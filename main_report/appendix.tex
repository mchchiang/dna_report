\section{Additional Simulation Details}
We discuss additional simulation details here for those who intend to reproduce the results obtained in the report.

\subsection{The Velocity-Verlet Integration Scheme}
\label{app:velocity-verlet}
The velocity-Verlet integration scheme is a variation of the Verlet algorithm. The Verlet algorithm can be obtained by considering the Taylor expansion of the particle's position $\bm{r}(t)$ around $t$:
\begin{align}
\bm{r}(t + \inc t) & = \bm{r}(t) + \bm{v}(t) \inc t + \frac{\bm{a}(t) }{2}\inc t^2 + \O{\inc t^3}\\
\bm{r}(t - \inc t) & =  \bm{r}(t) - \bm{v}(t) \inc t  + \frac{\bm{a}(t)}{2} \inc t^2 + \O{\inc t^3},
\end{align}
where $\bm{v}{t}$ and $\bm{a}(t)$ are the velocity and the acceleration of the particle respectively. Subtracting the two equations gives:
\begin{equation}
\bm{r}(t+\inc t) - \bm{r}(t - \inc t) = 2 \
\end{equation}

\subsection{Mapping between Simulation and Physical Timescales}
In section~\ref{sec:mapping}, we give a simple outline of how to convert between simulation and physical units. In particular, we derived a relation 

One needs to recognise that there are in fact three relevant timescales when performing molecular dynamics simulations. Firstly, the length scale $\sigma$, the mass $m$, and the energy $\hat{\epsilon}$ give rise to a natural simulation unit called the Lennard Jones time unit :
\begin{eqnarray}
\tau_{LJ} = \sqrt{\frac{m\sigma}{\hat{\epsilon}}}
\end{eqnarray}

\section{Additional Figures}
\subsection{Reproducing Results Reported by Dodd \etal}
\label{app:doddresult}


\section{Accessing the Source Code}
\label{app:code}

